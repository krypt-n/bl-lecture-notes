Die wichtigsten Eigenschaften einer Beschreibungslogik sind
\emph{Ausdrucksstärke} und \emph{Komplexität}. Ausdruckstärke kann man
nicht linear quantifizieren, sondern nur beschreiben und
charakterisieren.

\subsection{Bisimulation}\label{bisimulation}

\subsubsection{Definition}\label{definition-2}

Seien $I_{1}$ und $I_{2}$ Interpretationen. Relation
$\rho \subseteq \Delta^{I_{1}} \times \Delta^{I_{2}}$ ist Bisimulation
zwischen $I_{1}$ und $I_{2}$, wenn gilt:

\begin{enumerate}
\def\labelenumi{\arabic{enumi}.}
\item
  Wenn $d_{1}\text{\ $\rho$}\text{\ d}_{2}$, dann gilt für alle
  Konzeptnamen A: $d_{1} \in A^{I_{1}}$ gdw. $d_{2} \in A^{I_{2}}$.
\item
  Wenn $d_{1}\text{\ $\rho$}\text{\ d}_{2}$ und
  $\left( d_{1},d_{1}^{'} \right) \in r^{I_{1}}$ für beliebigen
  Rollennamen $r$, dann gibt es ein $d_{2}^{'} \in \Delta^{I_{2}}$
  mit ${d'}_{1}\text{\ $\rho$}{\ d'}_{2}$ und
  $\left( d_{2},d_{2}^{'} \right) \in r^{I_{2}}$.
\item
  Wenn $d_{1}\text{\ $\rho$}\text{\ d}_{2}$ und
  $\left( d_{2},d_{2}^{'} \right) \in r^{I_{2}}$ für beliebigen
  Rollennamen $r$, dann gibt es ein $d_{1}^{'} \in \Delta^{I_{1}}$
  mit ${d'}_{1}\text{\ $\rho$}{\ d'}_{2}$ und
  $\left( d_{1},d_{1}^{'} \right) \in r^{I_{1}}$.
\end{enumerate}

Seien $I_{1}$ und $I_{2}$ Interpretationen,
$d_{1} \in \Delta^{I_{1}}$, $d_{2} \in \Delta^{I_{2}}$:

$(I_{1},d_{1}) \sim (I_{2},d_{2})$: Es gibt Bisimulation $\rho$
zwischen $I_{1}$ und $I_{2}$ mit $d_{1}\text{\ $\rho$}\text{\ d}_{2}$.
Die leere Relation ist immer Bisimulation.

\subsubsection{Theorem 3.2}\label{theorem-3.2}

Seien $I_{1}$, $I_{2}$ Interpretationen,
$d_{1} \in \Delta^{I_{1}}$ und $d_{2} \in \Delta^{I_{2}}$. Wenn
$(I_{1},d_{1}) \sim (I_{2},d_{2})$, dann gilt für alle ALC-Konzepte
$C$:

$d_{1} \in C^{I_{1}}$ gdw. $d_{2} \in C^{I_{2}}$

Beweisskizze per Induktion über die Struktur von C. Sei $\rho$ eine
Bisimulation zwischen $I_{1}$ und $I_{2}$ mit
$d_{1}\text{\ $\rho$}\text{\ d}_{2}$.

\textbf{I.A.} $C = A$ ist Konzeptname. Nach Bedingung 1. der
Bisimulation gilt $d_{1} \in A^{I_{1}}$ gdw. $d_{2} \in A^{I_{2}}$.

\textbf{I.S.} Unterscheide Fälle gemäß des äußersten Konstruktes von C.
Es genügen $\neg$,$\sqcap$, $\exists\text{r.C}$:

\begin{enumerate}
\def\labelenumi{\arabic{enumi}.}
\item
  $C = \neg D$
\end{enumerate}

\begin{quote}
$d_{1} \in C^{I_{1}}$ gdw. $d_{1} \notin D^{I_{1}}$ (Semantik) gdw.
$d_{2} \notin D^{I_{2}}$ (I.V.) gdw. $d_{2} \in C^{I_{2}}$
(Semantik)
\end{quote}

\begin{enumerate}
\def\labelenumi{\arabic{enumi}.}
\item
  $C = D_{1} \sqcap D_{2}$
\end{enumerate}

\begin{quote}
$d_{1} \in C^{I_{1}}$ gdw. $d_{1} \in D_{1}^{I_{1}}$und
$d_{1} \in D_{2}^{I_{1}}$ (Semantik) gdw. $d_{2} \in D_{1}^{I_{2}}$
und $d_{2} \in D_{2}^{I_{2}}$ (I.V.) gdw. $d_{2} \in C^{I_{2}}$
(Semantik)
\end{quote}

\begin{enumerate}
\def\labelenumi{\arabic{enumi}.}
\item
  $C = \exists r.D$
\end{enumerate}

\begin{quote}
Hinrichtung und Rückrichtung analog über Semantik, 2. Bedingung der
Bisimulation, I.V., Semantik.
\end{quote}

\subsection{Ausdrucksstärke}\label{ausdrucksstuxe4rke}

\subsubsection{Definition}\label{definition-3}

Eine \emph{Eigenschaft} $E$ ist eine Menge von Paaren $(I,d)$, wobei
$I$ eine Interpretation und $d \in \Delta^{I}$ ein Element in $I$
ist. $E$ ist \emph{ausdrückbar in ALC}, wenn es ein ALC-Konzept $C$
gibt, so dass für alle $I$ und $d \in \Delta^{I}$ gilt:
$\left( I,\ d \right) \in E$ gdw. $d \in C^{I}$.

\subsubsection{Theorem 3.4}\label{theorem-3.4}

Die zusätzlichen Eigenschaften von ALCI und ALCQ sind in ALC nicht
ausdrückbar.

Beweisskizze. Finde Bisimulation für die dies nicht gilt.

\subsubsection{Theorem 3.5}\label{theorem-3.5}

Sei $E$ eine Eigenschaft. Wenn es Interpretation $I_{1}$, $I_{2}$
und Elemente $d_{1} \in \Delta^{I_{1}}$ und
$d_{2} \in \Delta^{I_{2}}$ gibt, so dass

\begin{itemize}
\item
  $\left( I_{1},d_{1} \right) \in E$ und
  $\left( I_{2},d_{2} \right) \in E$ sowie
\item
  $(I_{1},d_{1}) \sim (I_{2},d_{2})$
\end{itemize}

dann ist $E$ nicht in ALC ausdrückbar.

\subsubsection{Theorem 3.6}\label{theorem-3.6}

Wenn ein ALC-Konzept $C$ bzgl. einer ALC-TBox $T$ erfüllbar ist,
dann haben $C$ und $T$ ein gemeinsames Baummodell $I$. $I$ Baum,
Wurzel in $C^{I}$.

\subsubsection{Unravelling}\label{unravelling}

Sei $I$ eine Interpretation und $d \in \Delta^{I}$. $d$-Pfad in
$I$: Sequenz $d_{0}d_{1}\ldots d_{n - 1}$, $n > 0$ mit

\begin{itemize}
\item
  $d_{0} = d$
\item
  für alle $i < n$: es gibt Rollenname $r$ mit
  $\left( d_{i},d_{i + 1} \right) \in r^{I}$.
\end{itemize}

Wir setzen
$\text{end}\left( d_{0}\ldots d_{n - 1} \right) = d_{n - 1}$.

Definition: Unravelling von $I$ an Stelle $d$ ist folgende
Interpretation $J$:

\begin{itemize}
\item
  $\Delta^{J} =$ Menge aller $d$-Pfade in $I$
\item
  $A^{J} = \left\{ p \in \Delta^{J}\  \right|\text{\ end}\left( p \right) \in A^{I}\}$
\item
  $r^{J} = \left\{ \left( p,p' \right) \in \Delta^{J} \times \Delta^{J}\ |\ \exists e:p^{'} = p \cdot e\ \mathrm{\text{und}}\ \left( \text{end}\left( p \right),e \right) \in r^{I} \right\}$
\end{itemize}

Erklärung: Erzeuge Knoten, die den Folgen entsprechen, füge sie den
Konzepten hinzu, die als letztes Element in der Folge vorkommen und
erzeuge Kanten die den weiterführenden Rollen entsprechen.

\hypertarget{lemma-3.8}{\subsubsection{Lemma 3.8}\label{lemma-3.8}}

Sei $J$ Unravelling von $I$ an Stelle $d$. Für alle ALC-Konzepte
$C$ und alle $p \in \Delta^{J}$ gilt:
$\text{end}\left( p \right) \in C^{I}$ gdw. $p \in C^{J}$.

Beweisskizze. Zeige Bisimulation
$\left( I,end\left( p \right) \right)\sim\left( J,p \right)$, z.B.
$\rho = \left\{ \left( \text{end}\left( p \right),p \right) \in \Delta^{I} \times \Delta^{J}\  \right|\text{\ p\ }\mathrm{\text{ist}}\text{\ d}\mathrm{- Pfad}\}\ $(Bilde
alle Knoten in $J$ auf ihr Ende ab).

\begin{enumerate}
\def\labelenumi{\arabic{enumi}.}
\item
  gilt per Definition von $J$.
\item
  Angenommen $e$,$\rho$, $p$ und
  $\left( e,e^{'} \right) \in r^{I}$. Dann $e = end\left( p \right)$
  per Konstruktion von $J$. Wegen $\left( e,e^{'} \right) \in r^{I}$
  ist $pe'$ Pfad. Nach Konstruktion von $J$ gilt
  $\left( p,pe^{'} \right) \in r^{J}$.
\end{enumerate}

\subsubsection{Theorem 3.6}\label{theorem-3.6-1}

Wenn ein ALC-Konzept $C$ bzgl. einer TBox $T$ erfüllbar ist, dann
haben $C$ und $T$ ein gemeinsames Baummodell $I$.

Beweisskizze. Wurzel von Unravelling $J$ von $I$ an der Stelle
$d \in C^{I}$ nach \protect\hyperlink{lemma-3.8}{Lemma 3.8} in
$C^{J}$. Zu zeigen: $J$ ist Modell von $T$. Sei
$D \sqsubseteq E$ in $T$ und $p \in D^{J}$. Nach
\protect\hyperlink{lemma-3.8}{Lemma 3.8} ist dann
$\text{end}\left( p \right) \in D^{I}$ und weil $I$ Modell von $T$
folgt $\text{end}\left( p \right) \in E^{I}$. Mit
\protect\hyperlink{lemma-3.8}{Lemma 3.8} folgt
$p \in E^{J} \Rightarrow J \vDash D \sqsubseteq E$.

\subsubsection{Bisimulation versus
Ausdrucksstärke}\label{bisimulation-versus-ausdrucksstuxe4rke}

Bisimulation entspricht nicht der Ausdrucksstärke von ALC.

\subsubsection{Bisimulation in ALCI}\label{bisimulation-in-alci}

Füge 2 Regeln hinzu, sodass Vorgänger auch simuliert sein müssen.

\subsection{Ausdrucksstärke und
Modellkonstruktion}\label{ausdrucksstuxe4rke-und-modellkonstruktion}

\subsubsection{Größe von Konzepten und
TBoxen}\label{gruxf6uxdfe-von-konzepten-und-tboxen}

\emph{Größe} $\left| C \right|$ eines ALC-Konzeptes $C$ ist induktiv
definiert:

\begin{itemize}
\item
  $\left| A \right| = 1$
\item
  $\left| \neg C \right| = \left| C \right| + 1$
\item
  $\left| C \sqcap D \right| = \left| C \sqcup D \right| = \left| C \right| + \left| D \right| + 1$
\item
  $\left| \exists r.C \right| = \left| \forall r.C \right| = \left| C \right| + 3$
\end{itemize}

\emph{Größe} $\left| C \right|$ einer TBox $T$ ist

\begin{itemize}
\item
  $\sum_{C \sqsubseteq D \in T}^{}{\left| C \right| + \left| D \right| + 1}$
\end{itemize}

\subsubsection{Teilkonzepte}\label{teilkonzepte}

\begin{itemize}
\item
  $\text{sub}\left( C \right)$ ist Menge der Teilkonzepte von $C$
  (einschließlich $C$)
\item
  $\text{sub}\left( T \right) \bigcup_{C \sqsubseteq D \in T}^{}{\text{sub}\left( C \right) \cup sub\left( D \right)}$
\item
  $\text{sub}\left( C,T \right) sub\left( C \right) \cup sub\left( D \right)$
\end{itemize}

\hypertarget{lemma-3.13}{\subsubsection{Lemma 3.13}\label{lemma-3.13}}

$\left| \text{sub}\left( C,T \right) \right| \leq \left| C \right| + \left| T \right|$

\subsubsection{Typ}\label{typ}

Sei $I$ eine Interpretation, $d \in \Delta^{I}$. Der \emph{Typ}
$t_{I}\left( d \right)$ \emph{von} $d$ \emph{in} $I$ ist
$t_{I}\left( d \right) = \left\{ D \in sub\left( C,T \right)\  \right|\ d \in D^{I}\}$.

Erklärung: Alle Konzepte in $T$ und $C$, die ein Objekt $d$
erfüllt.

\hypertarget{lemma-3.15}{\subsubsection{Lemma 3.15}\label{lemma-3.15}}

Für jede Interpretation $I$ gilt:
$\#\left\{ t_{I}\left( d \right)\ |\ d \in \Delta^{I} \right\} \leq 2^{\left| C \right| + |T|}$

\subsubsection{Filtration}\label{filtration}

Sei $I$ Interpretation. Definiere Äquivalenzrelation $\sim$ auf
$\Delta^{I}$: $d \sim e$ gdw.
$t_{I}\left( d \right) = t_{I}\left( e \right)$. Wir bezeichnen diese
Äquivalenzklasse von $d \in \Delta^{I}$ bzgl. $\sim$ mit
$\left\lbrack d \right\rbrack$. Die Filtration von $I$ bzgl. $C$
und $T$ ist folgende Interpretation $J$:

\begin{itemize}
\item
  $\Delta^{J} = \left\{ \left\lbrack d \right\rbrack\ |\ d \in \Delta^{I} \right\}$
\item
  $A^{J} = \left\{ \left\lbrack d \right\rbrack\ |\ d \in A^{I} \right\}$
  für alle $A \in sub\left( C,T \right)$
\item
  $r^{J} = \left\{ \left( \left\lbrack d \right\rbrack,\left\lbrack e \right\rbrack \right)|\ \exists d^{'} \in \left\lbrack d \right\rbrack,\ e^{'} \in \left\lbrack e \right\rbrack:\left( d^{'},e^{'} \right) \in r^{I} \right\}$
  für alle Rollennamen $r$
\end{itemize}

\hypertarget{theorem-3.17}{\paragraph{Theorem 3.17}\label{theorem-3.17}}

Wenn $I$ Modell von $C$ und $T$, so auch $J$, bzw. für alle
$d \in \Delta^{I}$ und $D \in sub(C,T)$ gilt: $d \in D^{I}$ gdw.
$\left\lbrack d \right\rbrack \in D^{J}$

Beweisskizze per Induktion über die Struktur von D.

\textbf{I.A.} $C = A$ folgt aus Definition $A^{J}$.

\textbf{I.S.}

\begin{enumerate}
\def\labelenumi{\arabic{enumi}.}
\item
  $\neg$, $\sqcap$ einfach mittels Semantik und I.A.
\item
  $D = \exists r.E$

  \begin{enumerate}
  \def\labelenumii{\alph{enumii}.}
  \item
    Hinrichtung
  \end{enumerate}
\end{enumerate}

\begin{quote}
$d \in \left( \exists r.E \right)^{I} \Leftrightarrow$ (Semantik) es
gibt $e \in \Delta^{I}$ mit $\left( d,e \right) \in r^{I}$ und
$e \in E^{I} \Rightarrow$ (Definition $r^{J}$ und I.V.) es gibt
$e \in \Delta^{I}$ mit
$\left( \left\lbrack d \right\rbrack,\left\lbrack e \right\rbrack \right) \in r^{J}$
und $\left\lbrack e \right\rbrack \in E^{J} \Leftrightarrow$ (Semantik
$\exists$)
$\left\lbrack d \right\rbrack \in \left( \exists r.E \right)^{J}$
\end{quote}

\begin{enumerate}
\def\labelenumi{\alph{enumi}.}
\item
  Rückrichtung
\end{enumerate}

\begin{quote}
$\left\lbrack d \right\rbrack \in \left( \exists r.E \right)^{J} \Leftrightarrow$
(Semantik $\exists$) es gibt $e \in \Delta^{J}$ mit
$\left( \left\lbrack d \right\rbrack,\left\lbrack e \right\rbrack \right) \in r^{J}$
und $\left\lbrack e \right\rbrack \in E^{J} \Leftrightarrow$
(Definition $r^{J}$ und I.V.) es gibt $e \in \Delta^{J}$, es gibt
$d^{'} \in \left\lbrack d \right\rbrack$,
$e^{'} \in \left\lbrack e \right\rbrack$,
$\left( d^{'},\ e^{'} \right) \in r^{I}$ und
$e^{'} \in E^{I} \Rightarrow$ (Semantik $\exists$)
$d^{'} \in \left( \exists r.E \right)^{I} \Rightarrow$
($d \sim d^{'}$) $d \in \left( \exists r.E \right)^{I}$
\end{quote}

\subsubsection{Theorem 3.11 -- Beschränkte
Modelleigenschaft}\label{theorem-3.11-beschruxe4nkte-modelleigenschaft}

Wenn ein Konzept $C$ bzgl. einer $\text{TBox}$ $T$ erfüllbar ist,
dann haben $C$ und $T$ ein gemeinsames Modell der
\emph{Kardinalität} $\leq 2^{\left| C \right| + |T|}$.

Beweis. Folgt aus \protect\hyperlink{theorem-3.17}{Theorem 3.17} und
\protect\hyperlink{lemma-3.15}{Lemma 3.15}.

\subsubsection{Theorem 3.18}\label{theorem-3.18}

ALCQI hat nicht die endliche Modelleigenschaft.

Beweis: $A$ hat nur unendliche Modelle bzgl. folgender TBox:

\begin{itemize}
\item
  $\top \sqsubseteq \exists r.\neg A$
\item
  $T \sqsubseteq ( \leq 1\ r^{-}\ \top)$
\end{itemize}

\subsubsection{Theorem 3.11 --
Erfüllbarkeit}\label{theorem-3.11-erfuxfcllbarkeit}

Wenn $C$ erfüllbar bzgl. $T$, dann haben $C$ und $T$ Modell der
Größe $\leq 2^{\left| C \right| + \left| T \right|}$. Erfüllbarkeit
ist also entscheidbar (Erzeuge alle Interpretationen mit
$\left| \Delta \right|^{I} \leq 2^{n}$ und prüfe, ob ein Modell
darunter ist).

\subsubsection{Lemma 3.19}\label{lemma-3.19}

Gegeben sei ein Konzept $C$ und endliche Interpretation $I$. Man
kann in polynomieller Zeit -- genauer in Zeit
$O\left( \left| C \right| \cdot \left| \Delta^{I} \right| \right)$ --
die Extension $C^{I}$ berechnen.

Beweis: Rekursiver Algorithmus über die Definition der Konzeptsemantik.

\subsubsection{Korollar 3.20}\label{korollar-3.20}

Gegeben seien $C$, $T$ und endliche Interpretation $I$. Man kann
in polynomieller Zeit -- genauer: in Zeit
$O(\left( \left| T \right| + \left| C \right| \right) \cdot \left| \Delta^{I} \right|)$
-- entscheiden, ob $I$ ein Modell von $C$ und $T$ ist.

\subsubsection{Theorem 3.21}\label{theorem-3.21}

In ALC ist Erfüllbarkeit bzgl. TBoxen entscheidbar.
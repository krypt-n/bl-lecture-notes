\subsection{Wissensrepräsentation}\label{wissensrepruxe4sentation}

\subsubsection{Definition}\label{definition}

Entwicklung von Formalismen, mittels derer Wissen über die Welt in
abstrakter Weise beschrieben werden kann und die effektiv verwendet
werden können, um intelligente Anwendungen zu realisieren.

\subsubsection{Wohldefinierte Syntax und
Semantik}\label{wohldefinierte-syntax-und-semantik}

\begin{itemize}
\item
  Syntax: die Sprache, in der Wissen repräsentiert wird und hier stets
  symbolisch und logikbasiert
\item
  Semantik: fixiert die Bedeutung des repräsentierten Wissens in
  exakter, eindeutiger Weise

  \begin{itemize}
  \item
    Deklarative Semantik ist unabhängig von verarbeitender Software
  \end{itemize}
\end{itemize}

\subsubsection{In Beschreibungslogik}\label{in-beschreibungslogik}

\begin{itemize}
\item
  Beschränkung auf konzeptuelles Wissen -\textgreater{} Abstraktion
\item
  Schlussfolgern (explizit nach implizit) ist Mehrwert gegenüber
  Datenbanken

  \begin{itemize}
  \item
    Entscheidbarkeit und geringe Komplexität erwünscht
  \end{itemize}
\item
  Beschreibungslogiken sind Logikfamilie
\end{itemize}